% Options for packages loaded elsewhere
\PassOptionsToPackage{unicode}{hyperref}
\PassOptionsToPackage{hyphens}{url}
\documentclass[
]{article}
\usepackage{xcolor}
\usepackage[margin=1in]{geometry}
\usepackage{amsmath,amssymb}
\setcounter{secnumdepth}{-\maxdimen} % remove section numbering
\usepackage{iftex}
\ifPDFTeX
  \usepackage[T1]{fontenc}
  \usepackage[utf8]{inputenc}
  \usepackage{textcomp} % provide euro and other symbols
\else % if luatex or xetex
  \usepackage{unicode-math} % this also loads fontspec
  \defaultfontfeatures{Scale=MatchLowercase}
  \defaultfontfeatures[\rmfamily]{Ligatures=TeX,Scale=1}
\fi
\usepackage{lmodern}
\ifPDFTeX\else
  % xetex/luatex font selection
\fi
% Use upquote if available, for straight quotes in verbatim environments
\IfFileExists{upquote.sty}{\usepackage{upquote}}{}
\IfFileExists{microtype.sty}{% use microtype if available
  \usepackage[]{microtype}
  \UseMicrotypeSet[protrusion]{basicmath} % disable protrusion for tt fonts
}{}
\makeatletter
\@ifundefined{KOMAClassName}{% if non-KOMA class
  \IfFileExists{parskip.sty}{%
    \usepackage{parskip}
  }{% else
    \setlength{\parindent}{0pt}
    \setlength{\parskip}{6pt plus 2pt minus 1pt}}
}{% if KOMA class
  \KOMAoptions{parskip=half}}
\makeatother
\usepackage{color}
\usepackage{fancyvrb}
\newcommand{\VerbBar}{|}
\newcommand{\VERB}{\Verb[commandchars=\\\{\}]}
\DefineVerbatimEnvironment{Highlighting}{Verbatim}{commandchars=\\\{\}}
% Add ',fontsize=\small' for more characters per line
\usepackage{framed}
\definecolor{shadecolor}{RGB}{248,248,248}
\newenvironment{Shaded}{\begin{snugshade}}{\end{snugshade}}
\newcommand{\AlertTok}[1]{\textcolor[rgb]{0.94,0.16,0.16}{#1}}
\newcommand{\AnnotationTok}[1]{\textcolor[rgb]{0.56,0.35,0.01}{\textbf{\textit{#1}}}}
\newcommand{\AttributeTok}[1]{\textcolor[rgb]{0.13,0.29,0.53}{#1}}
\newcommand{\BaseNTok}[1]{\textcolor[rgb]{0.00,0.00,0.81}{#1}}
\newcommand{\BuiltInTok}[1]{#1}
\newcommand{\CharTok}[1]{\textcolor[rgb]{0.31,0.60,0.02}{#1}}
\newcommand{\CommentTok}[1]{\textcolor[rgb]{0.56,0.35,0.01}{\textit{#1}}}
\newcommand{\CommentVarTok}[1]{\textcolor[rgb]{0.56,0.35,0.01}{\textbf{\textit{#1}}}}
\newcommand{\ConstantTok}[1]{\textcolor[rgb]{0.56,0.35,0.01}{#1}}
\newcommand{\ControlFlowTok}[1]{\textcolor[rgb]{0.13,0.29,0.53}{\textbf{#1}}}
\newcommand{\DataTypeTok}[1]{\textcolor[rgb]{0.13,0.29,0.53}{#1}}
\newcommand{\DecValTok}[1]{\textcolor[rgb]{0.00,0.00,0.81}{#1}}
\newcommand{\DocumentationTok}[1]{\textcolor[rgb]{0.56,0.35,0.01}{\textbf{\textit{#1}}}}
\newcommand{\ErrorTok}[1]{\textcolor[rgb]{0.64,0.00,0.00}{\textbf{#1}}}
\newcommand{\ExtensionTok}[1]{#1}
\newcommand{\FloatTok}[1]{\textcolor[rgb]{0.00,0.00,0.81}{#1}}
\newcommand{\FunctionTok}[1]{\textcolor[rgb]{0.13,0.29,0.53}{\textbf{#1}}}
\newcommand{\ImportTok}[1]{#1}
\newcommand{\InformationTok}[1]{\textcolor[rgb]{0.56,0.35,0.01}{\textbf{\textit{#1}}}}
\newcommand{\KeywordTok}[1]{\textcolor[rgb]{0.13,0.29,0.53}{\textbf{#1}}}
\newcommand{\NormalTok}[1]{#1}
\newcommand{\OperatorTok}[1]{\textcolor[rgb]{0.81,0.36,0.00}{\textbf{#1}}}
\newcommand{\OtherTok}[1]{\textcolor[rgb]{0.56,0.35,0.01}{#1}}
\newcommand{\PreprocessorTok}[1]{\textcolor[rgb]{0.56,0.35,0.01}{\textit{#1}}}
\newcommand{\RegionMarkerTok}[1]{#1}
\newcommand{\SpecialCharTok}[1]{\textcolor[rgb]{0.81,0.36,0.00}{\textbf{#1}}}
\newcommand{\SpecialStringTok}[1]{\textcolor[rgb]{0.31,0.60,0.02}{#1}}
\newcommand{\StringTok}[1]{\textcolor[rgb]{0.31,0.60,0.02}{#1}}
\newcommand{\VariableTok}[1]{\textcolor[rgb]{0.00,0.00,0.00}{#1}}
\newcommand{\VerbatimStringTok}[1]{\textcolor[rgb]{0.31,0.60,0.02}{#1}}
\newcommand{\WarningTok}[1]{\textcolor[rgb]{0.56,0.35,0.01}{\textbf{\textit{#1}}}}
\usepackage{graphicx}
\makeatletter
\newsavebox\pandoc@box
\newcommand*\pandocbounded[1]{% scales image to fit in text height/width
  \sbox\pandoc@box{#1}%
  \Gscale@div\@tempa{\textheight}{\dimexpr\ht\pandoc@box+\dp\pandoc@box\relax}%
  \Gscale@div\@tempb{\linewidth}{\wd\pandoc@box}%
  \ifdim\@tempb\p@<\@tempa\p@\let\@tempa\@tempb\fi% select the smaller of both
  \ifdim\@tempa\p@<\p@\scalebox{\@tempa}{\usebox\pandoc@box}%
  \else\usebox{\pandoc@box}%
  \fi%
}
% Set default figure placement to htbp
\def\fps@figure{htbp}
\makeatother
\setlength{\emergencystretch}{3em} % prevent overfull lines
\providecommand{\tightlist}{%
  \setlength{\itemsep}{0pt}\setlength{\parskip}{0pt}}
\usepackage{booktabs}
\usepackage{longtable}
\usepackage{array}
\usepackage{multirow}
\usepackage{wrapfig}
\usepackage{float}
\usepackage{colortbl}
\usepackage{pdflscape}
\usepackage{tabu}
\usepackage{threeparttable}
\usepackage{threeparttablex}
\usepackage[normalem]{ulem}
\usepackage{makecell}
\usepackage{xcolor}
\usepackage{bookmark}
\IfFileExists{xurl.sty}{\usepackage{xurl}}{} % add URL line breaks if available
\urlstyle{same}
\hypersetup{
  pdftitle={MECH476: Engineering Data Analysis in R},
  pdfauthor={Michael Thill},
  hidelinks,
  pdfcreator={LaTeX via pandoc}}

\title{MECH476: Engineering Data Analysis in R}
\usepackage{etoolbox}
\makeatletter
\providecommand{\subtitle}[1]{% add subtitle to \maketitle
  \apptocmd{\@title}{\par {\large #1 \par}}{}{}
}
\makeatother
\subtitle{Chapter 6 Homework: Strings, Dates, and Tidying}
\author{Michael Thill}
\date{11 December, 2025}

\begin{document}
\maketitle

\section{Chapter 6 Homework}\label{chapter-6-homework}

For this homework assignment, you will use data from Twitter that
include tweets (2011 to 2017) from Colorado senators, which can be
downloaded from Canvas. Just FYI---some tweets were cut off before
Twitter's character limit; just work with the data you have. The
original data are from
\href{https://github.com/fivethirtyeight/data/tree/master/twitter-ratio}{FiveThirtyEight}.

When a question asks you to make a plot, remember to set a theme, title,
subtitle, labels, colors, etc. It is up to you how to personalize your
plots, but put in some effort and think about making the plotting
approach consistent throughout the document. For example, you could use
the same theme for all plots. I also like to use the subtitle as a place
for the main summary for the viewer.

\newpage

\subsection{Question 1: Hashtags}\label{question-1-hashtags}

Within a pipeline using the Colorado-only tweet data, select
\texttt{text} variable and use \texttt{stringr::str\_extract\_all()}
with a pattern of
\texttt{"\#(\textbackslash{}\textbackslash{}d\textbar{}\textbackslash{}\textbackslash{}w)+"}
to extract all of the hashtags from the tweets. This will return a list
with one element. How many hashtags were used by Colorado senators?

\subsection{Question 2: Fires}\label{question-2-fires}

Colorado is on fire right now and has experienced many wildfires over
the years. Let's examine senators' tweet activity related to wildfires
based on hashtags. Using the character vector of hashtags you extracted
in Question 1, search for the hashtags that include ``fire'' or
``wildfire''. How many hashtags included ``fire''? How many included
``wildfire''?

\subsection{Question 3: Wildfires}\label{question-3-wildfires}

Now, let's look at general tweets concerning wildfires. First, subset
the data to a dataframe that includes tweets containing the word
``wildfire'' and their corresponding timestamp and user. Specifically,
(a) select \texttt{text}, \texttt{date}, and \texttt{user} and (b)
filter to text strings that include the word ``wildfire'' using
\texttt{dplyr::filter()} and \texttt{stringr::str\_detect()}.

\subsection{Question 4: Senators}\label{question-4-senators}

Which Colorado senator tweets more about wildfires?

\subsection{Question 5: Timing}\label{question-5-timing}

Using the same \texttt{wildfires} dataframe, create a summary table that
shows the number of tweets containing the word ``wildfire'' by year
(2011-2017). Which year has the most tweets about wildfires? Why might
this be the case? (Hint: Think about what happened in the previous
year.)

\newpage

\subsection{Question 6: Monthly tweets}\label{question-6-monthly-tweets}

Create a bar chart that answers the question: Are Colorado senators more
active at a certain time of year? Hints: Convert \texttt{month} to a
factor. Fill by \texttt{user}.

\newpage

\subsection{Question 7: Hourly tweets}\label{question-7-hourly-tweets}

Create a histogram of tweets by hour of day to visualize when our
senators are tweeting.

\newpage

\section{Appendix}\label{appendix}

\begin{Shaded}
\begin{Highlighting}[]
\CommentTok{\# set global options for figures, code, warnings, and messages}
\NormalTok{knitr}\SpecialCharTok{::}\NormalTok{opts\_chunk}\SpecialCharTok{$}\FunctionTok{set}\NormalTok{(}\AttributeTok{fig.width=}\DecValTok{6}\NormalTok{, }\AttributeTok{fig.height=}\DecValTok{4}\NormalTok{, }\AttributeTok{fig.path=}\StringTok{"../figs/"}\NormalTok{,}
                      \AttributeTok{echo=}\ConstantTok{FALSE}\NormalTok{, }\AttributeTok{warning=}\ConstantTok{FALSE}\NormalTok{, }\AttributeTok{message=}\ConstantTok{FALSE}\NormalTok{)}

\FunctionTok{library}\NormalTok{(tidyverse)}
\FunctionTok{library}\NormalTok{(lubridate)}
\FunctionTok{library}\NormalTok{(kableExtra)}


\NormalTok{twitter\_raw\_data }\OtherTok{\textless{}{-}} \FunctionTok{read\_csv}\NormalTok{(}\StringTok{"../data/senators.csv"}\NormalTok{)}

\NormalTok{tweets }\OtherTok{\textless{}{-}}\NormalTok{ twitter\_raw\_data }\SpecialCharTok{\%\textgreater{}\%}
  \FunctionTok{select}\NormalTok{(created\_at, text, user, state) }\SpecialCharTok{\%\textgreater{}\%}
  \FunctionTok{drop\_na}\NormalTok{() }\SpecialCharTok{\%\textgreater{}\%}
  \FunctionTok{rename}\NormalTok{(}\AttributeTok{date =}\NormalTok{ created\_at) }\SpecialCharTok{\%\textgreater{}\%}
  \FunctionTok{mutate}\NormalTok{(}\AttributeTok{text =} \FunctionTok{str\_to\_lower}\NormalTok{(text))}

\NormalTok{co\_tweets }\OtherTok{\textless{}{-}}\NormalTok{ tweets }\SpecialCharTok{\%\textgreater{}\%}
  \FunctionTok{filter}\NormalTok{(state }\SpecialCharTok{==} \StringTok{"CO"}\NormalTok{) }\SpecialCharTok{\%\textgreater{}\%}
  \FunctionTok{mutate}\NormalTok{(}
    \AttributeTok{date\_parsed =} \FunctionTok{mdy\_hm}\NormalTok{(date),}
    \AttributeTok{year\_num =} \FunctionTok{year}\NormalTok{(date\_parsed),}
    \AttributeTok{month\_name =} \FunctionTok{month}\NormalTok{(date\_parsed, }\AttributeTok{label =} \ConstantTok{TRUE}\NormalTok{, }\AttributeTok{abbr =} \ConstantTok{TRUE}\NormalTok{),}
    \AttributeTok{hour\_of\_day =} \FunctionTok{hour}\NormalTok{(date\_parsed)}
\NormalTok{  )}


\CommentTok{\# wildfire hashtag list}
\CommentTok{\# filter to tweets concerning wildfires}
\CommentTok{\# number of wildfire tweets by senator}
\CommentTok{\# number of wildfire tweets by year }
\CommentTok{\# create plot of tweets by month and user}
\CommentTok{\# create plot of cumulative hourly tweets by senator}
\end{Highlighting}
\end{Shaded}


\end{document}
